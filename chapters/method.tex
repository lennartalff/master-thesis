% !TeX root = ../Thesis.tex

\chapter{An Approach to Agile Maneuvering}
\label{sec:approach-to-agile-maneuvering}

\section{Overview}

\section{System Dynamics}
\label{sec:system-dynamics}

\subsection{Choice of Reference Frames}
\begin{figure}[h!]
	\centering
	\includegraphics[width=0.7\linewidth]{placeholders/reference_frames.png}
	\caption{Definition of the world-fixed inertial frame of reference $\mathcal{W}$ and the body-fixed frame $\mathcal{B}$.}
\end{figure}
\subsubsection{World-fixed Frame}
\subsubsection{Body-fixed Frame}

\subsection{Kinematics}

We write the combined velocity vector \nub as
\begin{equation}
	\nub = 
	\begin{bmatrix}
		\prescript{\mathcal{B}}{}{\vb} \\
		\prescript{\mathcal{B}}{}{\omeb}
	\end{bmatrix}
	,
\end{equation}
with 
\begin{equation}
	\label{eq:velocities-in-body-frame}
	\prescript{\mathcal{B}}{}{\vb} = 
	\begin{bmatrix}
		u & v & w
	\end{bmatrix}^\top
	\text{ and }
	\prescript{\mathcal{B}}{}{\omeb} = 
	\begin{bmatrix}
		p & q & r
	\end{bmatrix}^\top
	,
\end{equation}
where $\prescript{\mathcal{B}}{}{\vb}$ and $\prescript{\mathcal{B}}{}{\omeb}$ are the linear and angular velocity expressed in the body-fixed frame $\mathcal{B}$, respectively.

The pose of the vehicle in the world-fixed frame $\mathcal{W}$ reads

\begin{equation}
	\etab = 
	\begin{bmatrix}
		\prescript{\mathcal{W}}{}{\pbo} \\
		\prescript{\mathcal{W}}{}{\Theb_{\mathcal{W}\mathcal{B}}}
	\end{bmatrix}
	,
\end{equation}
with
\begin{equation}
	\prescript{\mathcal{W}}{}{\pbo} = 
	\begin{bmatrix}
		x & y & z
	\end{bmatrix}^\top
	\text{ and }
	\Theb_{\mathcal{W}\mathcal{B}} =
	\begin{bmatrix}
		\phi & \theta & \psi
	\end{bmatrix}^\top,
\end{equation}
where $\prescript{\mathcal{W}}{}{\pbo}$ is the vehicle's position, i.e. the position of body frame origin $O_\mathcal{B}$, with respect to the world frame origin $O_\mathcal{W}$, and $\Theb_{\mathcal{W}\mathcal{B}}$ denotes the euler angle vector describing the rotation between $\mathcal{W}$ and $\mathcal{B}$ following the extrinsic $x$-$y$-$z$ convention.

We write the relation between the linear and angular velocities $\vlinbody$ and $\vangbody$ expressed in the body-fixed frame and the corresponding velocities in the world-fixed frame as
\begin{equation}
	\label[]{eq:velocity-world-body-transformation}
	\begin{bmatrix}
		\vlinworld \\
		\vangworld
	\end{bmatrix}
	=
	\underbrace{
	\begin{bmatrix}
		\Rbodyworld & \bm{0}_{3 \times 3} \\
		\bm{0}_{3 \times 3} & \TbodyWorld
	\end{bmatrix}
	}_{\mbox{\scriptsize $\coloneqq \Jb_\Theta$}}
	\begin{bmatrix}
		\vlinbody \\
		\vangbody
	\end{bmatrix},
\end{equation}
with is the rotation matrix 
\begin{equation}
	\label{eq:rotation-matrix}
	\Rbodyworld = 
	\begin{bmatrix}
		\text{c}\psi\text{c}\theta
		& \text{c}\psi \text{s}\theta \text{s}\phi - \text{s}\psi \text{c}\phi
		& \text{s}\psi \text{s}\phi + \text{c}\psi \text{c}\phi \text{s} \theta \\
		\text{s}\psi \text{c}\theta
		& \text{c}\psi \text{c}\phi + \text{s}\phi \text{s}\theta \text{s}\psi
		& \text{s}\theta \text{s}\psi \text{c}\phi - \text{c}\psi \text{s}\phi \\
		-\text{s}\theta
		& \text{c}\theta \text{s}\phi
		& \text{c}\theta \text{c}\phi
	\end{bmatrix}
\end{equation}
and the transformation matrix
\begin{equation}
	\label{eq:transformation}
	\TbodyWorld = 
	\begin{bmatrix}
		1 & \text{s}\phi \text{t}\theta & \text{c}\phi \text{t}\theta \\
		0 & \text{c}\phi & -\text{s}\phi \\
		0 & \frac{\text{s}\phi}{\text{t}\theta} & \frac{\text{c}\phi}{\text{c}\theta}
	\end{bmatrix},
\end{equation}
where $\text{s}\cdot$, $\text{c}\cdot$ and $\text{t}\cdot$ represent the functions $\sin(\cdot)$, $\cos(\cdot)$ and $\tan(\cdot)$, respectively.

The time derivative of $\Rbodyworld$ is
\begin{equation}
	\label{eq:rotation-matrix-derivative}
	\prescript{\mathcal{W}}{}{\dot{\bm{R}}_{\mathcal{B}}} = \Rbodyworld \Sb(\vangbody)
	,
\end{equation}
with the cross product skew-symmetric matrix
\begin{equation}
	\Sb(\vangbody) = 
	\begin{bmatrix}
		0 & -\vangbodyz & \vangbodyy \\
		\vangbodyz & 0 & -\vangbodyx \\
		-\vangbodyy & \vangbodyx & 0
	\end{bmatrix},
	\quad
	\vangbody = 
	\begin{bmatrix}
		\vangbodyx \\
		\vangbodyy \\ 
		\vangbodyz
	\end{bmatrix}
	.
\end{equation}



\subsection{Equations of Motion}

\begin{itemize}
	\color{red}
	\item Start with the general eq of motion
	\item Apply and justify/explain the simplifications and assumptions to get to the simplified eom in \Cref{eq:equation-of-motion-translational,eq:equation-of-motion-rotational} \todo[inline]{further simplications for trajectory generation. but gazebo simulation uses these equations. how to structure this?}
	\item 
\end{itemize}

In the following, we derive a dynamic model for the HippoCampus \ac{uauv} based on \cite{Fossen11}.
Subsequently, we simplify the model by exploiting the properties of the robot and making reasonable assumptions.

Let
\begin{equation}
	\taub = 
	\begin{bmatrix}
		\prescript{\mathcal{B}}{}{\fb} \\
		\prescript{\mathcal{B}}{}{\mb}
	\end{bmatrix}
\end{equation}
denote the load vector with 
\begin{equation}
	\prescript{\mathcal{B}}{}{\fb} = 
	\begin{bmatrix}
		X & Y & Z
	\end{bmatrix}^\top
	\text{ and }
	\prescript{\mathcal{B}}{}{\mb} = 
	\begin{bmatrix}
		K & M & N
	\end{bmatrix}^\top
\end{equation}
being the forces and moments with respect to the body-fixed frame $\mathcal{B}$.

We write the rigid-body equation of motion as
\begin{equation}
	\label{eq:rigid-body-equation-of-motion}
	\Mrigid \nubp
	+ \Crigid(\nub) \nub
	= \taub
	,
\end{equation}
with
\begin{equation}
	\label{eq:rigid-body-mass-matrix}
	\Mrigid =
	\begin{bmatrix}
		m \Ib_{3 \times 3} & \bm{0} \\
		\bm{0} & \Jb
	\end{bmatrix}
\end{equation},
where $\Mrigid$ is the rigid-body mass matrix, $\Crigid$ is the rigid-body Coriolis and centripetal matrix, and $\Jb$ the vehicle's inertia matrix with respect to its center of gravity.

To account for hydrodynamic and hydrostatic effects, we extend \Cref{eq:rigid-body-equation-of-motion} by the corresponding terms, which yields in accordance with \cite{Fossen11}
\begin{equation}
	\label{eq:6dof-equation-of-motion}
	\Mrigid \nubp + \Crigid(\nub) \nub +
	\underbrace{
		\Madded \nubp +
		\Cadded(\nub) \nub +
		\Dadded(\nub) \nub
	}_\text{hydrodynamic loads}
	+ 
	\underbrace{
		\gb(\etab)
	}_\text{\makebox[0pt]{hydrostatic load}}
	= \taub
	,
\end{equation}
where $\Madded$ is the added mass matrix, $\Cadded$ the added Coriolis matrix, and $\Dadded$ the added damping matrix.
The hydrostatic load, denoted by $\gb(\etab)$, represents the forces acting on the body due to gravity and buoyancy and moments induced by them.

We make the following assumptions with respect to HippoCampus \ac{uauv} to simplify \Cref{eq:6dof-equation-of-motion}:
\begin{itemize}
	\item Symmetry with respect to $xz$, $yz$ and $xy$ planes.
	\item The center of gravity lies in the origin $O_\mathcal{B}$ of the body-fixed frame $\mathcal{B}$.
	\item The difference in the magnitudes of buoyancy force and gravitational force is zero, i.e. the vehicle is neutrally buoyant. Therefore, the restoring force is assumed to be zero.
	\item The center of buoyancy and the center of gravity coincide. The resulting restoring moment will be zero.
	\item The vehicle's velocity is relatively small.
	\item The motion of the vehicle is uncoupled.
	\item The principal axes of inertia coincide with the body-fixed frame axes.
\end{itemize}

Using the body symmetry, the added mass matrix becomes \todo{Fossen 7.5.2 p.172}
\begin{equation}
	\Madded =
	\text{diag}\left(X_{\up}, Y_{\vp}, Z_{\wpo}, K_{\pp}, M_{\qp}, N_{\rp}\right)
\end{equation}
and since the axes of inertia coincide with the body-fixed frame, the rigid-body inertia matrix $\Jb$ becomes diagonal as well, i.e.
\begin{equation}
	\Jb = \text{diag}(J_x, J_y, J_z)
	.
\end{equation}
For uncoupled motions \cite{Fossen11} suggests assuming diagonal shape for $\Dadded$ as well. \todo{find a reference for assuming linear terms only. see hastedt}


Since the vehicle is neutrally buoyant and the center of buoyancy and gravity are identical, the hydrostatic forces and moments vanish.

With $\Madded = \text{diag}\left(\Mvadded, \Jadded\right)$ and $\Dadded = \text{diag}(\Dvadded, \Domegaadded)$, we introduce the following auxiliary variables
\begin{equation}
	\Mbs = \left(m\Ib + \Mvadded\right),\text{ and }
	\Jbs = \left(\Jb + \Jadded\right)
\end{equation}
and write \Cref{eq:6dof-equation-of-motion} separated by the translational and rotational dynamics as
\begin{equation}
	\label{eq:equation-of-motion-translational}
	\Mbs \vlinbodydot = \vlinbody \times \Mbs \vangbody - \Dvadded \vlinbody - \gb(\etab) + u_1 \eb_1
\end{equation}\todo{Muss es nicht nur die obere Haelfte von $\gb$ sein? Muesste $\Mbs$ nicht vor $\vlinbody$ stehen?}
and
\begin{equation}
	\label{eq:equation-of-motion-rotational}
	\Jbs \vangbodydot =
	\vlinbody \times \Mbs \vangbody
	- \vangbody \times \Jbs \vangbody
	- \Domegaadded \vangbody
	- \gb(\etab)
	+ \begin{bmatrix}
		u_2 & u_3 & u_4	
	\end{bmatrix}^\top
	.
\end{equation}







Because \emph{principal axes of inertia} coincide with body frame axes, $\Jb$ will be diagonal, i.e. $\Jb = \text{diag}(J_x, J_y, J_z)$.


\todo{make sure to reference \cite{Fossen11}}
\begin{equation}
\Mb = \Mrigid + \Madded
\end{equation}
\todo[inline]{Replace \taub with $\inputbody$? Assumption of no other external loads?}

\begin{equation}
	\label{eq:input-vector}
	\inputbody =
	\begin{bmatrix}
		u_1 &
		0 &
		0 &
		u_2 &
		u_3 &
		u_4 
	\end{bmatrix}^\top
\end{equation}

\begin{equation}
	\Mbs = \left(m\Ib + \Mvadded\right), \text{ with }
	\Madded = \text{diag}\left(\Mvadded, \Jadded\right)
\end{equation}
\begin{equation}
	\Jbs = \left(\Jb + \Jadded\right), \text{ with }
	\Dadded = \text{diag}(\Dvadded, \Domegaadded)
\end{equation}
Splitting \Cref{eq:6dof-equation-of-motion} in a translational and rotational gives

respectively.

We make the following assumptions:

Using body symmetry for $xz$, $yz$ and $xy$ planes \todo{Fossen 7.5.5}


In general, 
\begin{equation}
	\Dadded(\nub) = -\text{diag}(X_\text{u}, Y_\text{v}, Z_\text{w}, K_\text{p}, M_\text{q}, N_\text{r})
\end{equation}
\todo{Nicht-linearen Teil lass ich direkt weg, weil der auch nicht mal simuliert wird.}


\begin{figure}[h!]
	\centering
	\includegraphics[width=0.7\linewidth]{placeholders/free_body.png}
	\caption{Free body diagram of the HippoCampus \mu AUV with buoyancy force $\bm{f}_\textrm{B}$, gravitational force $\bm{f}_\textrm{G}$, thruster forces $\bm{f}_\textrm{1:4}$, and thruster torques $\bm{m}_{1:4}$.}
\end{figure}
\begin{itemize}
	\color{red}
	\item Rigid body dynamics -> added hydrodynamic terms -> assumptions and simplifications -> yields simplified dynamic model
	\item Thruster model
	\begin{itemize}
		\item assume motor speed is reached instantaneously (fast in comparison to the body dynamics)
		\item assume quadratic thrust curve (Hastedt)
		\item neglect relative velocity
		\item neglect dead band
		\item neglect forward/reverse asymmetry.
	\end{itemize}
\end{itemize}

\subsection{Differential Flatness (Optional)}
{\color{red}
	show diff. flatness and motivate it by highlighting that it is useful for the trajectory-generation section
}

\section{Trajectory Generation}
\label{sec:trajectory-generation} 
We neglect the cross-coupling term in \Cref{eq:equation-of-motion-translational} and can write the translational equation of motion expressed in the world frame $\mathcal{W}$ as
\begin{equation}
	\label{eq:eom-world-without-cross-coupling}
	\Rbodyworld \, \Mbs \, \Rbodyworld^\top \, \pworldddot = \Rbodyworld\Dadded\Rbodyworld^\top \pworlddot + u_1 \exbodyinworld.
\end{equation}
We further assume the \ac{uauv} is mainly moving in surge direction, which seems reasonable when taking the thruster configuration into account. Therefore, we approximate the inertia and damping matrix by
\begin{equation}
	\label{eq:added-mass-trajectory-simplification}
	\Mbs = m\Ib + \text{diag}\left(X_{\dot{u}}, Y_{\dot{v}}, Z_{\dot{w}}\right)
	=
	m\Ib + X_{\dot{u}} \Ib
\end{equation}
and
\begin{equation}
	\label{eq:added-damping-trajectory-simplification}
	\Dvadded = -\text{diag}\left(X_\text{u}, Y_\text{v}, Z_\text{w}\right)
	= -X_\text{u} \Ib
\end{equation}

Using \Cref{eq:added-mass-trajectory-simplification,eq:added-damping-trajectory-simplification}, we rewrite \Cref{eq:eom-world-without-cross-coupling} as
\begin{equation}
	(m+X_{\dot{u}})\pworldddot
	=
	-X_u \pworlddot
	+ u_1\exbodyinworld
\end{equation}

and reordering yields
\begin{equation}
	\label{eq:trajectory-acceleration-eom}
	\pworldddot = \frac{\Rbodyworld \eb_1 f - X_u \pworlddot}{m + X_{\dot{u}}},
\end{equation}
where $f=u_1$ denotes the thrust input of the system.

The required thrust, given the acceleration $\pworldddot$ and the velocity $\pworlddot$, can be computed by applying the euclidean norm to \Cref{eq:trajectory-acceleration-eom}
\begin{equation}
	\label{eq:trajectory-thrust}
	f =
	\left\lVert
	\left(m + X_{\dot{u}}\right) \pworldddot + X_u \pworlddot
	\right\rVert.
\end{equation}

In the following all equations are expressed in the world-fixed frame $\mathcal{W}$ and the corresponding indices are omitted in favor of readability.

\begin{itemize}
	\color{red}
	\item make clear, that we do not care for any constrains (affine translational or input) for now. refer to the next section.
	\item Define the state variable consisting of translational variables and derivatives
\end{itemize}


\begin{equation}
	\label{eq:trajectory-state}
	\sbo = 
	\begin{bmatrix}
		\pbo & \vb & \ab	
	\end{bmatrix}
	\quad
	\text{and}
	\quad
	\sbp =
	\begin{bmatrix}
		\vb &
		\ab &
		\jb
	\end{bmatrix}
\end{equation}
with
\begin{equation}
	\vb = \pbp
	\quad
	\text{and}
	\quad
	\ab = \pbpp
	\quad
	\text{and}
	\quad
	\jb = \pbppp
\end{equation}
being the velocity, acceleration and jerk.

\begin{equation}
	\label{eq:trajectory-state-per-axis}
	\sbo_{i} =
	\begin{bmatrix}
		p_{i} &
		v_{i} &
		a_{i}
	\end{bmatrix}
\end{equation}

\begin{equation}
	\label{eq:trajectory-state-derivative-per-axis}
	\sbp_\text{i} =
	\begin{bmatrix}
		v_{i} &
		a_{i} &
		j_{i}
	\end{bmatrix}
\end{equation}

\subsection{Kinematic Derivation of the jerk-optimal Trajectory}
\begin{itemize}
	\color{red}
	\item Define the cost function
	\item derive the minimum jerk solution per axis
	\item 
\end{itemize}

\begin{equation}
	\label{eq:cost-function}
	J_\text{\Sigma} = \frac{1}{T} \int_{0}^{T}\left\lVert \jb(t)\right\rVert^2 \text{d}t
\end{equation}

\begin{equation}
	\label{eq:cost-function-per-axis}
	J_\text{\Sigma} = \sum_{i=1}^{3}J_\text{i}, \text{ where } J_{i} = \frac{1}{T}\int_{0}^{T}j_i(t)^2\text{d}t
\end{equation}

\begin{equation}
	\label{eq:optimal-trajectory-per-axis}
	\sbo^* = 
	\begin{bmatrix}
		\frac{\alpha}{120}t^5
		+ \frac{\beta}{24}t^4
		+ \frac{\gamma}{6}t^3
		+ \frac{a_0}{2}t^2
		+ v_0 t
		+ p_0
		\\
		\frac{\alpha}{24} t^4
		+ \frac{\beta}{6} t^3
		+ \frac{\gamma}{2} t^2
		+ a_0 t
		+ v_0
		\\
		\frac{\alpha}{6} t^3
		+ \frac{\beta}{2} t^2
		+ \gamma t
		+ a_0
	\end{bmatrix}
\end{equation}

\begin{equation}
	\label{eq:trajectory-coefficients}
	\begin{bmatrix}
		\alpha \\
		\beta \\
		\gamma
	\end{bmatrix}
	=
	\begin{bmatrix}
		720 & -360T & 60T^2 \\
		-360T & 168T^2 * -24T^3 \\
		60T^2 & -24T^3 & 3T^4

	\end{bmatrix}
	\begin{bmatrix}
		\Delta p \\
		\Delta v \\
		\Delta a
	\end{bmatrix}
\end{equation}

\begin{equation}
	\label{eq:trajectory-delta-state}
	\begin{bmatrix}
		\Delta p \\
		\Delta v \\
		\Delta a
	\end{bmatrix}
	= 
	\begin{bmatrix}
		p_\text{f} - p_0 - v_0 T - \frac{1}{2}a_0 T^2 \\
		v_\text{f} - v_0 - a_0 T \\
		a_\text{f} - a_0
	\end{bmatrix}
\end{equation}


\subsection{Sampling Strategy}
\begin{itemize}
	\color{red}
	\item how to choose the final state to reach the high level goal, i.e. catching the ring as proposed scenario.
	\begin{itemize}
		\item vehicle tip position inside the ring
		\item different final attitudes possible (sampling on a section of a sphere around final tip position)
		\item calculate actual vehicle's position based on that
	\end{itemize}
	\item higher variety of final states increases chance to generate a feasible solution -> refer to following section
	\item introduce additional rotated inertial system to specify axis components or let them unspecified
	\item stress out the difference to MuellerHehn15, due to velocity dependency.
\end{itemize}
\begin{figure}[h!]
	\centering
	\includegraphics[width=0.7\linewidth]{placeholders/generation_example.png}
	\caption{Generated trajectories for different final states and time horizons.}
\end{figure}
\section{Feasibility of Trajectories}
\label{sec:feasibility}



\subsection{Input Feasibility}

\begin{equation}
	\label{eq:thrust-limits}
	\unit[0]{N} \leq f_\text{min} \leq f \leq f_\text{max}
\end{equation}

\begin{equation}
	\label{eq:body-rates-limit}
	\left\lVert\omeb\right\rVert
	\leq
	\omega_\text{max}
\end{equation}

\subsubsection{Thrust Input Feasibility}
\begin{itemize}
	\color{red}
	\item cubic instead of quadratic function is to be solved
	\item stress out, that feasibility criterium is quite coarse -> feasible solution might be classified as indeterminable
	\item following equations are per axis. Indies are omitted if unambiguous.
\end{itemize}

\begin{align}
	\label{eq:thrust-feasibility-equivilency-max}
	\max_{t \in \mathcal{T}} f(t)^2
	&\leq
	f_\text{max}^2 \\
	\label{eq:thrust-feasibility-equivilency-min}
	\min_{t \in \mathcal{T}} f(t)^2
	&\geq
	f_\text{min}^2
\end{align}

By suqaring \Cref{eq:trajectory-thrust}, we can write the squared thrust in per-axis notation
\begin{equation}
	\label{eq:thrust-squared-per-axis}
	f^2 = 
	\left\lVert
	\left(m + X_{\dot{u}}\right) \pbpp + X_u \pbp
	\right\rVert^2
	= 
	\sum_{k=1}^{3}
	\left(\left(m + X_{\dot{u}}\right)\ppp_k + X_u \pp_k \right)^2
\end{equation}
and find the conservative thrust limits by summing the per axis extrema.
\begin{align}
	\label{eq:fmax-per-axis}
	\max_{t \in \mathcal{T}} f(t)^2
	&\leq
	\sum_{k=1}^{3}
	\max_{t \in \mathcal{T}}
	\left(
		\left(m + X_{\dot{u}}\right)
		\ppp_k
		+ X_u \pp_k
	\right)^2 \\
	\label{eq:fmin-per-axis}
	\min_{t \in \mathcal{T}} f(t)^2
	&\geq
	\sum_{k=1}^{3}
	\min_{t \in \mathcal{T}}
	\left(
		\left(m + X_{\dot{u}}\right)
		\ppp_k
		+ X_u \pp_k
	\right)^2
\end{align}

We denote the maximum and minimum of the per axis thrust as \fmaxaxis and \fminaxis, respectively\todo{Hier ist im Unterschied zu Mueller eine kubische Funktion fuer die Nullstellen zu loesen} 
and write the extrema in \Cref{eq:fmax-per-axis,eq:fmax-per-axis} as
\begin{align}
	\max_{t \in \mathcal{T}}
	\left(
		\left(m + X_{\dot{u}}\right)
		\ppp_k
		+ X_u \pp_k
	\right)^2
	&= 
	\max\left\{\fmaxaxis^2, \fminaxis^2\right\} \\
	\min_{t \in \mathcal{T}}
	\left(
		\left(m + X_{\dot{u}}\right)
		\ppp_k
		+ X_u \pp_k
	\right)^2
	&=
	\begin{cases}
		\min\left\{\fmaxaxis^2, \fminaxis^2\right\}
		& \text{if } \fmaxaxis\cdot\fminaxis > 0 \\
		0
		& \text{otherwise}
	\end{cases}
\end{align}

A sufficient criterion, rendering the trajectory infeasible, is
\begin{equation}
	\label{eq:sufficient-infeasible-thrust}
	\max\left\{\fmaxaxis^2, \fminaxis^2\right\}
	> \fmaxsquare
	,
\end{equation}
whereas a sufficient criterion for feasibility with respect to the thrust input is if both
\begin{align}
	\label{eq:sufficient-feasible-thrust-max}
	\sum_{k=1}^{3}
	\max_{t \in \mathcal{T}}
	\left\{\fmaxaxis^2, \fminaxis^2\right\}
	& \leq
	\fmaxsquare \\
	\label{eq:sufficient-feasible-thrust-min}
	\sum_{k=1}^{3}
	\min_{t \in \mathcal{T}}
	\left\{\fmaxaxis^2, \fminaxis^2\right\}
	& \geq
	\fminsquare
\end{align}
hold. Note that it might be the case, that neither \Cref{eq:sufficient-infeasible-thrust} nor \Cref{eq:sufficient-feasible-thrust-max,eq:sufficient-feasible-thrust-min} hold. In this case, the interval $\mathcal{T}$ is divided in half and the feasibility is tested recursively on both half intervals. This procedure is repeated until either at least one interval is infeasible, rendering the whole trajectory infeasible, all tested sub intervals yield feasibility, or the tested sub intervals become smaller than some defined minimum interval $\Delta\tau_{\text{min}}$. In the latter case the trajectory is declared indeterminable.


\subsubsection{Body Rates Input Feasibility}
\begin{itemize}
	\color{red}
	\item Insert dynamics into definition of jerk
\end{itemize}

In terms of the jerk \pbppp, the vehicle dynamics are 
\begin{equation}
	\label{eq:jerk}
	\pbppp = 
	\frac{1}{m + X_{\dot{u}}}
	\left(
		\Rb \eb_1 \fp 
		+ \Rb \Sb(\omeb)\eb_1 f
		- X_u \pbpp
	\right)
\end{equation}

By squaring \Cref{eq:trajectory-thrust} and applying the first derivative, we get
\begin{equation}
	f\fp = 
	\left(
		\left(m + X_{\dot{u}}\right) \pbpp
		+ X_u \pbp
	\right)^\top
	\left(
		\left(m + X_{\dot{u}}\right) \pbppp
		+ X_u \pbpp
	\right).
\end{equation}
With \Cref*{eq:trajectory-acceleration-eom} this yields
\begin{equation}
	\label{eq:fdot}
	\fp = \left(\Rb \eb_1\right)^\top
	\left(
		\left(m + X_{\dot{u}}\right) \pbppp
		+ X_u \pbpp
	\right)
\end{equation}
for the derivative of the thrust.

We substitute \fp by \Cref{eq:fdot} in \Cref{eq:jerk} and simplify the equation
\begin{equation}
	\pbppp = 
	\frac{1}{m + X_{\up}}
	\left(
		\begin{bmatrix}
			1 & 0 & 0 \\
			0 & 0 & 0 \\
			0 & 0 & 0
		\end{bmatrix}
		\left(
			\left(m + X_{\dot{u}}\right)
			\pbppp
			+ X_u \pbpp
		\right)
		+
		\Rb
		\begin{bmatrix}
			0 \\
			\omega_3 \\
			-\omega_2
		\end{bmatrix}
		f
		- X_u \pbpp
	\right)
\end{equation}

\begin{equation}
	\label{eq:jerk-vs-bodyrates}
	\begin{bmatrix}
		0 & 0 & 0\\
		0 & 1 & 0\\
		0 & 0 & 1
	\end{bmatrix}
	\pbppp = 
	\Rb
	\begin{bmatrix}
		0 \\
		\omega_3 \\
		- \omega_2
	\end{bmatrix}
	\frac{f}{m + X_{\dot{u}}}
	-
	\begin{bmatrix}
		0 & 0 & 0 \\
		0 & 1 & 0 \\
		0 & 0 & 1
	\end{bmatrix}
	\frac{X_u}{m + X_{\dot{u}}}
	\pbpp
\end{equation}
We can observe, that the angular velocity about the roll axis $\omega_1$ has no influence on the translational dynamics. For the further course of this thesis we assume $\omega_1 = \unit[0]{\frac{rad}{s}}$.

To find a criterion for input feasibility with respect to the body rates, we reorder \Cref{eq:jerk-vs-bodyrates} for the body rates
\begin{equation}
	\begin{bmatrix}
		0 \\
		\omega_3 \\
		- \omega_2
	\end{bmatrix}
	=
	\frac{m + X_{\dot{u}}}{f}
	\Rb^\top
	\begin{bmatrix}
		0 & 0 & 0 \\
		0 & 1 & 0 \\
		0 & 0 & 1
	\end{bmatrix}
	\left(
		\pbppp + X_u \pbpp
	\right),
\end{equation}
apply the vector induced euclidean norm
\begin{equation}
	\left\lVert\Rb\right\rVert \leq 1,
\end{equation}
and get the bound of the body rates 
\begin{equation}
	\label{eq:body-rates-bound}
	\omega_3^2 + \omega_2^2 \leq
	\frac{
		\left(
			m + X_{\dot{u}}
		\right)^2
	}{f^2}
	\left\lVert
		\pbppp + X_u \pbpp
	\right\rVert^2
	.
\end{equation}

A conservative upper bound of \Cref{eq:body-rates-bound} in per axis notation denoted as $\bar{\omega}^2$ can be written with \Cref{eq:thrust-squared-per-axis} as
\begin{equation}
	\omega_2^2 + \omega_3^2
	\leq
	\bar{\omega}^2
	=
	\left(
		m + X_{\dot{u}}
	\right)^2
	\frac{
		\sum_{k=1}^{3}
		\max_{t \in \mathcal{T}}
		\left(
			\dddot{p}_k + X_u \ddot{p}
		\right)^2
	}{
		\sum_{k=1}^{3}
		\min_{t \in \mathcal{T}}
		\left(
			\left(m + X_{\dot{u}}\right)\ppp_k
			+ X_u \pp_k
		\right)^2
	}
	,
\end{equation}

and simplified with \Cref{eq:sufficient-feasible-thrust-min} and defining $\xi_k = \dddot{p}_k + X_u \ddot{p}$
\begin{equation}
	\bar{\omega}^2 =
	\left(
		m + X_{\dot{u}}
	\right)^2
	\frac{
		\sum_{k=1}^{3}
		\max_{t \in \mathcal{T}}
		\xi_k^2
	}{
		\sum_{k=1}^{3}
		\min_{t \in \mathcal{T}}
		\left\{\fmaxaxis^2, \fminaxis^2\right\}
	}
\end{equation}
The minimum and maximum of $\xi_k$ is denoted as \ximaxaxis and \ximinaxis, respectively. With
\begin{equation}
	\max_{t \in \mathcal{T}}
	\left(
		\dddot{p}_k + X_u \ddot{p}
	\right)^2 =
	\max_{t \in \mathcal{T}}
	\left\{
		\ximaxaxis^2, \ximinaxis^2
	\right\}
	,
\end{equation}
we formulate the feasibility criterion with respect to the body rates
\begin{equation}
	\label{eq:body-rates-criterion-per-axis}
	\bar{\omega}^2
	=
	\left(
		m + X_{\dot{u}}
	\right)^2
	\frac{
		\sum_{k=1}^{3}
		\max_{t \in \mathcal{T}}
		\left\{
			\ximaxaxis^2, \ximinaxis^2
		\right\}
	}{
		\sum_{k=1}^{3}
		\min_{t \in \mathcal{T}}
		\left\{\fmaxaxis^2, \fminaxis^2\right\}
	}
	\leq
	\omegamax
	.
\end{equation}
If \Cref{eq:body-rates-criterion-per-axis} does not hold, the tested section is split in half. The feasibility criterion is applied recursively on both sub intervals.


\subsection{Position Feasibility}

\section{Obstacle/Collision Avoidance}
\begin{itemize}
	\color{red}
	\item Make clear, that obstacle avoidance is not built into the trajectory generation itself. 
	\item can be seen as subsequent feasibility check. 
\end{itemize}

\section{Control}
\label{sec:control}
\begin{itemize}
	\color{red}
	\item one could use the min jerk trajectory approach simply for generation -> feed-back control to keep the vehicle on track
	\item computational efficiency and real-time capability of approach enables trajectory generation to be used for implicit feedback. Regenerate trajectories in each control time step.
	\item body rates will change rather slowly compared to quadrocopters -> instead of using body rates as in MuellerHehn15, use target attitude and attitude control (actually the body rates are derived from a target attitude anyway)
\end{itemize}

\section{Implementation}
\label{sec:implementation}
