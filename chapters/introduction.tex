% !TeX root = ../Thesis.tex
\chapter{Introduction}\label{chap:introduction}

\section{Motivation}
Maritime environmental field exploration and monitoring are among the most promising task to be addressed by autonomous underwater vehicles (AUVs).
The recent technological progress on the miniaturization of electronic components literally boosted the development of a new field of small-scale underwater robots.
While full-scale underwater robots are mostly deployed in ocean- and open water-like settings, this new class of small-scale vehicles enables accessing new \textit{confined} scenarios.
These environments are characterized by length-scales of tenths of meters and are often cluttered. 
This requires agile maneuvering and quick reactions by the robot’s planning algorithm if obstacles are detected.
Examples include the range from harbor areas and marinas towards industry tanks such as nuclear storage ponds in various forms, as summarized in \cite{Watson2020}.
%These environments come with strict space constraints and are often unstructured and cluttered. This requires agile maneuvering and quick reactions by the robot’s planning algorithm if obstacles are detected.
These confined settings pose strong constraints on the robot vehicles with regard to maneuverability and size.

This motivated the introduction of the class of micro AUVs ($\mu$AUVs) which has become a widely used category for tether-free underwater robots with a characteristic length up to 50\,cm \cite{micro-auv}.
Prominent examples include the AVEXIS \cite{Griffiths2016} and the HippoCampus platform \cite{duecker2020hippocampusx} as well as the LoCo AUV \cite{Edge2020}.
Recently, the term \textit{hydrobatics} has been established in a series of publications \cite{DueckerEtAl18, Duecker20, duecker2021aerobatics} as the marine counterpart of aerobatic aerial drones.
Specifically, hydrobatics refers to the class of underactuated underwater robots with mostly streamlined designs, i.\,e. torpedo-shaped, that are capable of agile maneuvering.
As a quantitative metric  -- analogue to their aerial siblings -- high vehicle motion speeds of multiple body lengths per second are used as a baseline for this categorization \cite{duecker2021aerobatics}.

Given the strict constraints on the vehicle size, the onboard computational and sensory resources of these vehicles are naturally limited \cite{Watson2020}.
Hence, a key design criterion for any algorithm to be developed is being computationally light-weight. 
Following this thought, $\mu$AUVs pose many different requirements than marine robots designed for the open sea, as these focus on long planning horizons, own slow dynamics, and carry sophisticated sensing equipment.

Closing the loop from the original tasks of field exploration and monitoring, both constitute high-level planning tasks that are usually performed with slow update rates due to their strong computational requirements.
Their output is often realized as a sequence of way points.
On the other end of the system pipeline, we see high-rate low-level control schemes that drive the vehicle attitude toward its desired orientation.
A desirable approach is, thus, to design a trajectory generation module that acts as an interface between both ends, taking, for instance, way points from the high-level planner as input while providing feasible setpoints at a high rate for the low-level controller. 
Moreover, these trajectory generation modules ideally consider the dynamic capabilities of the underwater robot while at the same time considering potentially present obstacles and avoiding those at a planning stage.
It is worth noting that these modules underlie strong restrictions with respect to the available computational resource as they have to be implemented embedded onboard the robot.
At the same time, they need to be \textit{real-time} capable. 
Note that throughout this thesis, we consider a slightly relaxed interpretation of being real-time capable, meaning being fast enough to keep up with the update rate. 
Hence, no real-time guarantees in worst-case execution time.





\section{Problem Statement}
Consider the scenario of a highly agile and underactuated \ac{uauv} maneuvering in a confined environment such as a small water basin.
Its mission goal is to reach a time-dependent goal state.
Such a time-dependent goal state can be represented by a moving ring, that is to be caught by the \ac{uauv}.
Additionally, there might be obstacles present, which make a method for collision avoidance mandatory.
Due to the physical limitations of wireless underwater communication \cite{Bettale08p1,GeistEtAl16}, all algorithms have to be executed onboard and in real-time to allow for \textit{fully} autonomous maneuvering.

From this scenario, we can derive the following requirements for a trajectory generation and tracking system:
\begin{itemize}
    \item real-time capability of the trajectory generation algorithm, when executed by the on-board hardware
    \item verification of the trajectories' feasibility based on the dynamics of the \ac{uauv}
    \item a control scheme to track the planned trajectory
    \item dynamic collision avoidance due to obstacles
\end{itemize}

The goal of this thesis is to develop such a system meeting the above-mentioned requirements and assess its performance by carrying out simulations and lab experiments.

\begin{figure}
    \centering
    \includesvg{agile_maneuvering_motivation}
    \caption{A \ac{uauv} agile maneuvering in a confined and cluttered underwater environment. The mission goal is to catch the moving hoop (\textit{red)}.}
    \label{fig:agile_maneuvering_motivation}
\end{figure}


\section{Contribution}
This thesis bridges several existing gaps present in current research on agile small-scale underwater robots. While there has been much recent development in self-localization methods for low-cost \acp{uauv}, these newly-won navigation capabilities have not yet been leveraged. The next logical step is the deployment of sophisticated path planning and trajectory tracking algorithms on these vehicles. % Notwendig 
A detailed literature review in \Cref{sec:review-motion-planning} on existing underwater path planning approaches in general and methods focusing purely on aggressive motion planning, e.\,g. in the field of aerial drones, revealed the need for agile maneuvering methods suitable for the underwater domain. 

From this, the present thesis exploits the synergies between aerobatic \acp{uav} and hydrobatic \acp{uauv}:  the concept of computational efficient trajectory generation for quadcopters in \cite{MuellerHehn15} is transferred to the underwater domain.
However, the differences in modeling the vehicle's dynamics have to be considered and addressed by the proposed planning method with respect to (i) computational efficiency, (ii) robustness of the trajectory tracking performance, and (iii) the sampling strategy. %\todo{Beachten wir die sampling strategie irgendwo?}
%\todo[inline]{Schon mal sagen, welche Erkenntnisse konkret aus den Experimenten gewonnen werden. Muss abgestimmt werden, mit dem, was ich am Ende tatsaechlich dort stehen habe}
Besides the domain transfer, this thesis proposes a new implementation architecture using the recently released framework ROS2 that greatly increases robustness and simplifies implementation.
The performance of the proposed method is demonstrated in a series of numerical and real-world experiments.


\section{Thesis Outline}
The subsequent structure of this thesis is as follows. 
\Cref{chap:fundamentals} provide a concise overview on the fundamentals for this work.
First, we present the micro underwater robot platform in \Cref{sec:hippo-platform} and point-out relevant details on its hardware and software concept. 
Subsequently, a review on state-of-the-art motion planning methods is conducted in \Cref{sec:review-motion-planning}.
This survey starts from methods used in the underwater domain but also opens its view to methods for agile path planning in other domains such as the very active field of aerial drones.
The identified relevant literature is summarized for the reader's convenience in \Cref{tab:state_of_the_art}.
%
Based on these fundamentals, we develop a novel approach to agile underwater motion planning in \Cref{chap:approach-to-agile-maneuvering}. 
Starting from the conceptual overview in \cref{sec:concept}, we study the system dynamics of the HippoCampus \ac{uauv} in \Cref{sec:system-dynamics}.
This is followed by the presentation of the sampling-based trajectory generation approach in \Cref{sec:trajectory-generation} which is elaborated in  \Cref{sec:feasibility}  and  \Cref{sec:collision-avoidance} with methods for feasibility checks and an approach for collision avoidance, respectively.
Moreover, we develop suitable control strategies in \Cref{sec:control}.
Finally, we provide insights on the actual implementation of the proposed agile planning framework in \Cref{sec:implementation}.
%
A detailed performance analysis is conducted in \Cref{chap:analysis}. It considers numerical studies and experimental studies.
%
The thesis concludes in \Cref{chap:conclusion} with a summary of the thesis' main findings and an outlook toward future work.



% The dynamic model of the HippoCampus \ac{uauv} is derived in \Cref{sec:system-dynamics} and applied to the sampling-based trajectory generation framework presented in \Cref{sec:trajectory-generation} and \Cref{sec:feasibility}.
% It is complemented by a attitude controller as proposed in \Cref{sec:control}.
% The framework's implementation details are presented in this thesis are given in \Cref{sec:implementation}




